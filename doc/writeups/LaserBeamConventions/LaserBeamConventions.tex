% arara: pdflatex

% start with vim --server latex %
\documentclass[]{article}

\usepackage{siunitx}
\usepackage{physics}
\usepackage{graphicx}
\usepackage{fullpage}

\author{C.D. Clark III}
\title{Laser beam propagation and the various conventions}

\begin{document}
\maketitle

\section{Introduction}

The equations used to determine beam diameter at range are based on Gaussian Beam Propagation Theory (GBP).
In physics textbooks, the GBP equations are formulated in terms of the beam \emph{radius} $\omega$, which
is the $1/\text{e}$ radius for the \textbf{field amplitude}, and the $1/\text{e}^2$ radius for the \textbf{field intensity}, which is proportional
to irradiance. The ANSI laser safety community uses the $1/e$ \emph{diameter} for the \textbf{irradiance}.
Siegman suggested using the ``second moment width'', defined to be two times the variance ($W = 2\sigma$),
because it obeys a universal propagation law. The variance is from the irradiance profile of the beam:
\begin{equation}
    \sigma = \frac{\int_{-\infty}^{\infty} (x-x_0)^2 I(x,y) \dd x \dd y}{ \int_{-\infty}^{\infty} I(x,y) \dd x \dd y} .
\end{equation}
Note that for non-circular beams, a width in the $x$ and $y$ directions can be defined.
Then the second moment width will popagate according to the usual range equation:
\begin{equation}
    W(z) = \sqrt{W_0^2 + M^2 \left(\frac{\lambda}{\pi W_0}\right)^2 (z - z_0)^2}.
\end{equation}
Apparently this equation holds for arbitrary non-twisted beams (i.e. non-Gaussian). For Gaussian beams, the second moment width
is the same as the $1/\text{e}^2$ radius.

The ISO standard defines ``beam diameter'' to be the $D4\sigma$ width, which is twice Siegman's width and will be the same as
the $1/\text{e}^2$ \emph{diameter} for a Gaussian beam.

Additionally, physics textbooks usually define the divergence to be the angle between the line defined by the beam
radius as a function of range and the center of the beam, i.e. the \emph{half-angle} defined by the $1/\text{e}^2$ points, and use
the symbol $\theta$. The ANSI laser safety community defines the divergence to be the \emph{full-angle} defined by the $1/\text{e}$ points, and uses the symbol, $\phi$.

Adding confusion is the fact that some sources may use $\Theta$ to
denote the full-angle divergence defined by the $1/\textbf{e}^2$ points, but this is not
universal. Newport's website for example has a page on Gaussian Optics (https://www.newport.com/n/gaussian-beam-optics) that uses $\theta$ as the full-width
divergence, defined by the $1/\text{e}^2$ points. In this document.

In terms of divergence, the range equation can be written as
\begin{equation}
    W(z) = \sqrt{W_0^2 + \Theta^(z - z_0)^2}.
\end{equation}
For the second moment width, the divergence is $M^2 \frac{\lambda}{\pi W_0}$, where $\frac{\lambda}{\pi W_0}$ is the diffraction-limited divergence and $M^2$ is beam quality factor.
However, for the other beam conventions, it is different.


The library implements the various beam propagation equations and supports multiple conventions.
It is easy to make errors when converting between the different beam diameter and divergence conventions, so this write-up serves as a reference.

\section{Conventions}

Consider three common conventions: GBP, ANSI laser safety, and ISO standard. We define the following symbols:


GBP Convention:

\begin{tabular}{rp{5in}}
    $\omega$ & $1/\text{e}^2$ beam radius. \\
    $\omega_0$& $1/e^2$ beam waist radius. \\
    $\theta$ & $1/e^2$ half-angle divergence. \\
    $z$ & some position along the beam axis measured with respect to a reference point. \\
    $z_0$ & position of the beam waist. $z$ is often assumed to be measured w.r.t. the beam waist position in physics textbooks so that $z_0 = 0$.
\end{tabular}

ANSI Laser Safety Convention (Based on ANSI Z136.4-2010):

\begin{tabular}{rp{5in}}
$D_L$ & $1/e$ beam diameter. \\
$D_W$ & $1/e$ beam waist diameter. \\
$\phi$ & $1/e$ full-angle divergence. \\
$r$ & range, i.e. some position along the beam axis measured with respect to a reference point, usually taken to be the aperture of the laser. \\
$r_0$ & position of the beam waist.
\end{tabular}

ISO $D4\sigma$ Convention:

\begin{tabular}{rp{5in}}
    $D4\sigma$ & four times beam variance, which is $1/\text{e}^2$ diameter for Gaussian beams. \\
    $D4\sigma_0$ & $1/e$ beam waist diameter. \\
    $\Theta$ & $D4\sigma$ full-angle divergence, which is $1/\text{e}^2$ divergence for Gaussian beams. \\
    $z$ & range, i.e. some position along the beam axis measured with respect to a reference point, usually taken to be the aperture of the laser. \\
    $z_0$ & position of the beam waist.
\end{tabular}


\section{Conversions}


The $1/\textbf{e}^2$ radius/diameter is $\sqrt{2}$ times larger than the $1/\text{e}$ radius/diameter. Therefore:
\begin{align}
\omega &= \frac{D4\sigma}{2} = \frac{D_L}{\sqrt{2}} \\
D_L &= \sqrt{2}\omega = \frac{D4\sigma}{\sqrt{2}} \\
D4\sigma &= \sqrt{2} D_L = 2\omega \\
\end{align}

We can derive the divergence for each convention in two different ways.
At positions far from the beam waist, the divergence can be computed using geometry/trigonometry. The small angle approximation
is typically assumed:

\begin{align}
    \tan\theta &\approx \theta = \frac{\omega}{z-z_0} \\
    \tan\phi &\approx \phi = \frac{D_L}{r-r_0} \\
    \tan\Theta &\approx \Theta = \frac{D4\sigma}{z-z_0}.
\end{align}
Since $r-r_0 = z-z_0$, this means that the different divergences are related to each other in the same that the differnt widths are related. For example
\begin{align}
    \Theta = \frac{D4\sigma}{z-z_0} = \frac{2\omega}{z-z_0} = 2\theta
\end{align}
So,
\begin{align}
\theta &= \frac{\Theta}{2} = \frac{\phi}{\sqrt{2}} \\
\phi &= \sqrt{2}\theta = \frac{\Theta}{\sqrt{2}} \\
\Theta &= \sqrt{2} \phi = 2\theta
\end{align}
Do derive each divergence in terms of its width, we start with the second moment divergence, which is the $1/\text{e}^2$ half-angle divergence for a Gaussian beam,
\begin{align}
    \theta &= M^2 \frac{\lambda}{\pi \omega_0} \\
    \frac{\Theta}{2}&= M^2 \frac{\lambda}{\pi  D4\sigma/2} = M^2 \frac{2\lambda}{\pi D4\sigma} \\
    \Theta&= M^2 \frac{4\lambda}{\pi D4\sigma} \\
\end{align}
and
\begin{align}
    \theta &= M^2 \frac{\lambda}{\pi \omega_0} \\
    \frac{\phi}{\sqrt{2}} &= M^2 \frac{\lambda}{\pi  D_L/\sqrt{2}} = M^2 \frac{\sqrt{2}\lambda}{\pi D_L} \\
    \phi &= M^2 \frac{2\lambda}{\pi D4\sigma} \\
\end{align}

We can also derive the relations from the range equations.

\begin{align}
    \omega^2 &= \omega_0^2 + M^4 \left(\frac{\lambda}{\pi \omega_0}\right)^2 (z-z_0)^2 \\
    4\omega^2 &= (2 \omega)^2 = D4\sigma^2 = 4\omega_0^2 + 4M^4 \left(\frac{\lambda}{\pi \omega_0}\right)^2 (z-z_0)^2 \\
              &= D4\sigma_0^2 + M^4 \left(\frac{2\lambda}{\pi D4\sigma_0 /2}\right)^2 (z-z_0)^2 \\
              &= D4\sigma_0^2 + M^4 \left(\frac{4\lambda}{\pi D4\sigma_0}\right)^2 (z-z_0)^2 \\
              &=  D4\sigma_0^2 + \Theta^2(z-z_0)^2 \\
    \Theta &= \frac{4\lambda}{\pi D4\sigma_0}
\end{align}
and
\begin{align}
    \omega^2 &= \omega_0^2 + M^4 \left(\frac{\lambda}{\pi \omega_0}\right) (z-z_0)^2 \\
    2\omega^2 &= (\sqrt{2}{ \omega})^2  = D_L^2 = 2\omega_0^2 + 2M^4 \left(\frac{\lambda}{\pi \omega_0}^2\right) (z-z_0)^2 \\
              &= (\sqrt{2}{ \omega})^2  = D_L^2 = D_0^2 + M^4 \left(\frac{\sqrt{2}\lambda}{\pi D_0/\sqrt{2}}^2\right) (z-z_0)^2 \\
              &= (\sqrt{2}{ \omega})^2  = D_L^2 = D_0^2 + M^4 \left(\frac{2\lambda}{\pi D_0}^2\right) (z-z_0)^2 \\
              &=  D_0^2 + \phi^2(z-z_0)^2 \\
    \phi &= \frac{2\lambda}{\pi D_0}
\end{align}


\end{document}
